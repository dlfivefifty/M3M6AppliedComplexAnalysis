\documentclass[12pt,a4paper]{article}

\usepackage[a4paper,text={16.5cm,25.2cm},centering]{geometry}
\usepackage{lmodern}
\usepackage{amssymb,amsmath}
\usepackage{bm}
\usepackage{graphicx}
\usepackage{microtype}
\usepackage{hyperref}
\setlength{\parindent}{0pt}
\setlength{\parskip}{1.2ex}

\hypersetup
       {   pdfauthor = { Sheehan Olver },
           pdftitle={ foo },
           colorlinks=TRUE,
           linkcolor=black,
           citecolor=blue,
           urlcolor=blue
       }




\usepackage{upquote}
\usepackage{listings}
\usepackage{xcolor}
\lstset{
    basicstyle=\ttfamily\footnotesize,
    upquote=true,
    breaklines=true,
    breakindent=0pt,
    keepspaces=true,
    showspaces=false,
    columns=fullflexible,
    showtabs=false,
    showstringspaces=false,
    escapeinside={(*@}{@*)},
    extendedchars=true,
}
\newcommand{\HLJLt}[1]{#1}
\newcommand{\HLJLw}[1]{#1}
\newcommand{\HLJLe}[1]{#1}
\newcommand{\HLJLeB}[1]{#1}
\newcommand{\HLJLo}[1]{#1}
\newcommand{\HLJLk}[1]{\textcolor[RGB]{148,91,176}{\textbf{#1}}}
\newcommand{\HLJLkc}[1]{\textcolor[RGB]{59,151,46}{\textit{#1}}}
\newcommand{\HLJLkd}[1]{\textcolor[RGB]{214,102,97}{\textit{#1}}}
\newcommand{\HLJLkn}[1]{\textcolor[RGB]{148,91,176}{\textbf{#1}}}
\newcommand{\HLJLkp}[1]{\textcolor[RGB]{148,91,176}{\textbf{#1}}}
\newcommand{\HLJLkr}[1]{\textcolor[RGB]{148,91,176}{\textbf{#1}}}
\newcommand{\HLJLkt}[1]{\textcolor[RGB]{148,91,176}{\textbf{#1}}}
\newcommand{\HLJLn}[1]{#1}
\newcommand{\HLJLna}[1]{#1}
\newcommand{\HLJLnb}[1]{#1}
\newcommand{\HLJLnbp}[1]{#1}
\newcommand{\HLJLnc}[1]{#1}
\newcommand{\HLJLncB}[1]{#1}
\newcommand{\HLJLnd}[1]{\textcolor[RGB]{214,102,97}{#1}}
\newcommand{\HLJLne}[1]{#1}
\newcommand{\HLJLneB}[1]{#1}
\newcommand{\HLJLnf}[1]{\textcolor[RGB]{66,102,213}{#1}}
\newcommand{\HLJLnfm}[1]{\textcolor[RGB]{66,102,213}{#1}}
\newcommand{\HLJLnp}[1]{#1}
\newcommand{\HLJLnl}[1]{#1}
\newcommand{\HLJLnn}[1]{#1}
\newcommand{\HLJLno}[1]{#1}
\newcommand{\HLJLnt}[1]{#1}
\newcommand{\HLJLnv}[1]{#1}
\newcommand{\HLJLnvc}[1]{#1}
\newcommand{\HLJLnvg}[1]{#1}
\newcommand{\HLJLnvi}[1]{#1}
\newcommand{\HLJLnvm}[1]{#1}
\newcommand{\HLJLl}[1]{#1}
\newcommand{\HLJLld}[1]{\textcolor[RGB]{148,91,176}{\textit{#1}}}
\newcommand{\HLJLs}[1]{\textcolor[RGB]{201,61,57}{#1}}
\newcommand{\HLJLsa}[1]{\textcolor[RGB]{201,61,57}{#1}}
\newcommand{\HLJLsb}[1]{\textcolor[RGB]{201,61,57}{#1}}
\newcommand{\HLJLsc}[1]{\textcolor[RGB]{201,61,57}{#1}}
\newcommand{\HLJLsd}[1]{\textcolor[RGB]{201,61,57}{#1}}
\newcommand{\HLJLsdB}[1]{\textcolor[RGB]{201,61,57}{#1}}
\newcommand{\HLJLsdC}[1]{\textcolor[RGB]{201,61,57}{#1}}
\newcommand{\HLJLse}[1]{\textcolor[RGB]{59,151,46}{#1}}
\newcommand{\HLJLsh}[1]{\textcolor[RGB]{201,61,57}{#1}}
\newcommand{\HLJLsi}[1]{#1}
\newcommand{\HLJLso}[1]{\textcolor[RGB]{201,61,57}{#1}}
\newcommand{\HLJLsr}[1]{\textcolor[RGB]{201,61,57}{#1}}
\newcommand{\HLJLss}[1]{\textcolor[RGB]{201,61,57}{#1}}
\newcommand{\HLJLssB}[1]{\textcolor[RGB]{201,61,57}{#1}}
\newcommand{\HLJLnB}[1]{\textcolor[RGB]{59,151,46}{#1}}
\newcommand{\HLJLnbB}[1]{\textcolor[RGB]{59,151,46}{#1}}
\newcommand{\HLJLnfB}[1]{\textcolor[RGB]{59,151,46}{#1}}
\newcommand{\HLJLnh}[1]{\textcolor[RGB]{59,151,46}{#1}}
\newcommand{\HLJLni}[1]{\textcolor[RGB]{59,151,46}{#1}}
\newcommand{\HLJLnil}[1]{\textcolor[RGB]{59,151,46}{#1}}
\newcommand{\HLJLnoB}[1]{\textcolor[RGB]{59,151,46}{#1}}
\newcommand{\HLJLoB}[1]{\textcolor[RGB]{102,102,102}{\textbf{#1}}}
\newcommand{\HLJLow}[1]{\textcolor[RGB]{102,102,102}{\textbf{#1}}}
\newcommand{\HLJLp}[1]{#1}
\newcommand{\HLJLc}[1]{\textcolor[RGB]{153,153,119}{\textit{#1}}}
\newcommand{\HLJLch}[1]{\textcolor[RGB]{153,153,119}{\textit{#1}}}
\newcommand{\HLJLcm}[1]{\textcolor[RGB]{153,153,119}{\textit{#1}}}
\newcommand{\HLJLcp}[1]{\textcolor[RGB]{153,153,119}{\textit{#1}}}
\newcommand{\HLJLcpB}[1]{\textcolor[RGB]{153,153,119}{\textit{#1}}}
\newcommand{\HLJLcs}[1]{\textcolor[RGB]{153,153,119}{\textit{#1}}}
\newcommand{\HLJLcsB}[1]{\textcolor[RGB]{153,153,119}{\textit{#1}}}
\newcommand{\HLJLg}[1]{#1}
\newcommand{\HLJLgd}[1]{#1}
\newcommand{\HLJLge}[1]{#1}
\newcommand{\HLJLgeB}[1]{#1}
\newcommand{\HLJLgh}[1]{#1}
\newcommand{\HLJLgi}[1]{#1}
\newcommand{\HLJLgo}[1]{#1}
\newcommand{\HLJLgp}[1]{#1}
\newcommand{\HLJLgs}[1]{#1}
\newcommand{\HLJLgsB}[1]{#1}
\newcommand{\HLJLgt}[1]{#1}



\def\qqand{\qquad\hbox{and}\qquad}
\def\qqfor{\qquad\hbox{for}\qquad}
\def\D{ {\rm d} }
\def\I{ {\rm i} }
\def\E{ {\rm e} }
\def\C{ {\mathbb C} }
\def\R{ {\mathbb R} }
\def\CC{ {\cal C} }
\def\HH{ {\cal H} }
\def\vc#1{ {\mathbf #1} }
\def\bbC{ {\mathbb C} }

\def\qqqquad{\qquad\qquad}
\def\qqfor{\qquad\hbox{for}\qquad}
\def\qqwhere{\qquad\hbox{where}\qquad}
\def\Res_#1{\underset{#1}{\rm Res}}\,
\def\sech{ {\rm sech}\, }



\def\Xint#1{ \mathchoice
   {\XXint\displaystyle\textstyle{#1} }%
   {\XXint\textstyle\scriptstyle{#1} }%
   {\XXint\scriptstyle\scriptscriptstyle{#1} }%
   {\XXint\scriptscriptstyle\scriptscriptstyle{#1} }%
   \!\int}
\def\XXint#1#2#3{ {\setbox0=\hbox{$#1{#2#3}{\int}$}
     \vcenter{\hbox{$#2#3$}}\kern-.5\wd0} }
\def\ddashint{\Xint=}
\def\dashint{\Xint-}


\def\addtab#1={#1\;&=}
\def\ccr{\\\addtab}
\def\ip<#1>{\left\langle{#1}\right\rangle}
\def\dx{\D x}
\def\dt{\D t}
\def\dz{\D z}

\def\norm#1{\left\| #1 \right\|}

\def\abs#1{\left|{#1}\right|}
\def\fpr(#1){\!\pr({#1})}

\def\sopmatrix#1{ \begin{pmatrix}#1\end{pmatrix} }

\def\endash{–}

\begin{document}

** M3M6: Methods of Mathematical Physics **

Dr. Sheehan Olver <br> s.olver@imperial.ac.uk

\section{Lecture 2: Cauchy's theorem}
\subsection{Complex-differentiable functions}
\textbf{Definition (Complex-differentiable)} Let $D \subset {\mathbb C}$ be an open set.  A function $f : D \rightarrow {\mathbb C}$ is called \emph{complex-differentiable} at a point $z_0 \in D$ if 

\[
   f'(z_0) =  \lim_{z \rightarrow z_0} {f(z) - f(z_0) \over z - z_0}
\]
exists, for any angle of approach to $z_0$.

\subsection{Holomorphic functions}
\textbf{Definition (Holomorphic)} Let $D \subset {\mathbb C}$ be an open set.  A function $f : D \rightarrow {\mathbb C}$ is called \emph{holomorphic} in $D$ if it is complex-differentiable at all $z \in D$.

\textbf{Definition (Entire)} A function is \emph{entire} if it is holomorphic in ${\mathbb C}$

\emph{Examples}

\begin{itemize}
\item[1. ] \[
1
\]
is entire


\item[2. ] \[
z
\]
is entire


\item[3. ] \[
1/z
\]
is holomorphic in ${\mathbb C} \backslash \{0\}$


\item[4. ] \[
\sin z
\]
is entire


\item[5. ] \[
\csc z
\]
is holomorphic in ${\mathbb C} \backslash \{\ldots,-2\pi,-\pi,0,\pi,2\pi,\ldots\}$


\item[6. ] \[
\sqrt z
\]
is holomorphic in ${\mathbb C} \backslash (-\infty,0]$

\end{itemize}
We can usually infer the domain where a function is holomorphic from a phase portrait, here we see that ${\rm arcsinh}\, z$ has cuts on $[\I,\I \infty)$ and $[-\I,-\I \infty)$, and a zero  (red\ensuremath{\endash}green\ensuremath{\endash}blue\ensuremath{\endash}red) at zero, hence we can infer that it is holomorphic in $\C \backslash ([\I,\I \infty) \cup [-\I,-\I \infty))$.


\begin{lstlisting}
(*@\HLJLk{using}@*) (*@\HLJLn{Plots}@*)(*@\HLJLp{,}@*) (*@\HLJLn{ComplexPhasePortrait}@*)(*@\HLJLp{,}@*) (*@\HLJLn{SpecialFunctions}@*)(*@\HLJLp{,}@*) (*@\HLJLn{ApproxFun}@*)
(*@\HLJLnf{phaseplot}@*)(*@\HLJLp{(}@*)(*@\HLJLoB{-}@*)(*@\HLJLnfB{4..4}@*)(*@\HLJLp{,}@*) (*@\HLJLoB{-}@*)(*@\HLJLnfB{4..4}@*)(*@\HLJLp{,}@*) (*@\HLJLn{z}@*) (*@\HLJLoB{->}@*) (*@\HLJLnf{asinh}@*)(*@\HLJLp{(}@*)(*@\HLJLn{z}@*)(*@\HLJLp{))}@*)
\end{lstlisting}

\includegraphics[width=\linewidth]{figures/Lecture2_1_1.pdf}

The following example $\sqrt{z-1} \sqrt{z+1}$ is analytic in ${\mathbb C}\backslash [-1,1]$ and will be returned to:


\begin{lstlisting}
(*@\HLJLnf{phaseplot}@*)(*@\HLJLp{(}@*)(*@\HLJLoB{-}@*)(*@\HLJLnfB{4..4}@*)(*@\HLJLp{,}@*) (*@\HLJLoB{-}@*)(*@\HLJLnfB{4..4}@*)(*@\HLJLp{,}@*) (*@\HLJLn{z}@*) (*@\HLJLoB{->}@*) (*@\HLJLnf{sqrt}@*)(*@\HLJLp{(}@*)(*@\HLJLn{z}@*)(*@\HLJLoB{-}@*)(*@\HLJLni{1}@*)(*@\HLJLp{)}@*)(*@\HLJLnf{sqrt}@*)(*@\HLJLp{(}@*)(*@\HLJLn{z}@*)(*@\HLJLoB{+}@*)(*@\HLJLni{1}@*)(*@\HLJLp{))}@*)
\end{lstlisting}

\includegraphics[width=\linewidth]{figures/Lecture2_2_1.pdf}

\subsection{Contours}
\textbf{Definition (Contour)} A \emph{contour}  is a continuous \& piecewise-continuously differentiable function $\gamma : [a,b] \rightarrow {\mathbb C}$.

\textbf{Definition (Simple)} A \emph{simple contour} is a contour that is 1-to-1.

\textbf{Definition (Closed)} A \emph{closed contour} is a contour such that $\gamma(a) = \gamma(b)$

\emph{Examples of contours}

\begin{itemize}
\item[1. ] Line segment $[a,b]$ is a simple contour, with $\gamma(t) =  t$


\item[2. ] Arc from $re^{ia}$ to $re^{ib}$ is a simple contour, with $\gamma(t) =  re^{i t}$


\item[3. ] Circle of radius $r$ is a closed simple contour, with $\gamma(t) = re^{i t}$ and $a = -\pi$, $b = \pi$


\item[4. ] \[
\gamma(t) = \cos (t+i)^2
\]
defines a contour that is not simple or closed


\item[5. ] \[
\gamma(t) = e^{i t} + e^{2i t}
\]
for $[a,b] = [-\pi,\pi]$ defines a contour that is closed but not simple

\end{itemize}
Here's an example of a closed contour that is not simple:


\begin{lstlisting}
(*@\HLJLn{a}@*)(*@\HLJLp{,}@*)(*@\HLJLn{b}@*) (*@\HLJLoB{=}@*) (*@\HLJLoB{-}@*)(*@\HLJLn{\ensuremath{\pi}}@*)(*@\HLJLp{,}@*) (*@\HLJLn{\ensuremath{\pi}}@*)
(*@\HLJLn{tt}@*) (*@\HLJLoB{=}@*) (*@\HLJLnf{range}@*)(*@\HLJLp{(}@*)(*@\HLJLn{a}@*)(*@\HLJLp{,}@*) (*@\HLJLn{stop}@*)(*@\HLJLoB{=}@*)(*@\HLJLn{b}@*)(*@\HLJLp{,}@*) (*@\HLJLn{length}@*)(*@\HLJLoB{=}@*)(*@\HLJLni{1000}@*)(*@\HLJLp{)}@*)

(*@\HLJLn{\ensuremath{\gamma}}@*) (*@\HLJLoB{=}@*) (*@\HLJLn{t}@*) (*@\HLJLoB{->}@*) (*@\HLJLnf{exp}@*)(*@\HLJLp{(}@*)(*@\HLJLn{im}@*)(*@\HLJLoB{*}@*)(*@\HLJLn{t}@*)(*@\HLJLp{)}@*) (*@\HLJLoB{+}@*)(*@\HLJLnf{exp}@*)(*@\HLJLp{(}@*)(*@\HLJLni{2}@*)(*@\HLJLn{im}@*)(*@\HLJLoB{*}@*)(*@\HLJLn{t}@*)(*@\HLJLp{)}@*)

(*@\HLJLnf{plot}@*)(*@\HLJLp{(}@*)(*@\HLJLn{real}@*)(*@\HLJLoB{.}@*)(*@\HLJLp{(}@*)(*@\HLJLn{\ensuremath{\gamma}}@*)(*@\HLJLoB{.}@*)(*@\HLJLp{(}@*)(*@\HLJLn{tt}@*)(*@\HLJLp{)),}@*) (*@\HLJLn{imag}@*)(*@\HLJLoB{.}@*)(*@\HLJLp{(}@*)(*@\HLJLn{\ensuremath{\gamma}}@*)(*@\HLJLoB{.}@*)(*@\HLJLp{(}@*)(*@\HLJLn{tt}@*)(*@\HLJLp{));}@*) (*@\HLJLn{ratio}@*)(*@\HLJLoB{=}@*)(*@\HLJLnfB{1.0}@*)(*@\HLJLp{,}@*) (*@\HLJLn{legend}@*)(*@\HLJLoB{=}@*)(*@\HLJLkc{false}@*)(*@\HLJLp{,}@*) (*@\HLJLn{arrow}@*)(*@\HLJLoB{=}@*)(*@\HLJLkc{true}@*)(*@\HLJLp{)}@*)
\end{lstlisting}

\includegraphics[width=\linewidth]{figures/Lecture2_3_1.pdf}

\subsection{Contour integrals}
\textbf{Definition (Contour integral)} The \emph{contour integral} over $\gamma$ is defined by

\[
\int_\gamma f(z) dz := \int_a^b f(\gamma(t)) \gamma'(t) dt
\]

\begin{lstlisting}
(*@\HLJLn{f}@*) (*@\HLJLoB{=}@*) (*@\HLJLnf{Fun}@*)(*@\HLJLp{(}@*) (*@\HLJLn{z}@*) (*@\HLJLoB{->}@*) (*@\HLJLnf{real}@*)(*@\HLJLp{(}@*)(*@\HLJLnf{exp}@*)(*@\HLJLp{(}@*)(*@\HLJLn{z}@*)(*@\HLJLp{)),}@*) (*@\HLJLnf{Arc}@*)(*@\HLJLp{(}@*)(*@\HLJLnfB{0.}@*)(*@\HLJLp{,}@*)(*@\HLJLnfB{1.}@*)(*@\HLJLp{,(}@*)(*@\HLJLni{0}@*)(*@\HLJLp{,}@*)(*@\HLJLn{\ensuremath{\pi}}@*)(*@\HLJLoB{/}@*)(*@\HLJLni{2}@*)(*@\HLJLp{)))}@*)  (*@\HLJLcs{{\#}}@*) (*@\HLJLcs{Not}@*) (*@\HLJLcs{holomorphic!}@*)

(*@\HLJLnf{plot}@*)(*@\HLJLp{(}@*)(*@\HLJLnf{domain}@*)(*@\HLJLp{(}@*)(*@\HLJLn{f}@*)(*@\HLJLp{);}@*) (*@\HLJLn{legend}@*)(*@\HLJLoB{=}@*)(*@\HLJLkc{false}@*)(*@\HLJLp{,}@*) (*@\HLJLn{ratio}@*)(*@\HLJLoB{=}@*)(*@\HLJLnfB{1.0}@*)(*@\HLJLp{,}@*) (*@\HLJLn{arrow}@*)(*@\HLJLoB{=}@*)(*@\HLJLkc{true}@*)(*@\HLJLp{)}@*)
\end{lstlisting}

\includegraphics[width=\linewidth]{figures/Lecture2_4_1.pdf}

\begin{lstlisting}
(*@\HLJLnf{sum}@*)(*@\HLJLp{(}@*)(*@\HLJLn{f}@*)(*@\HLJLp{)}@*)  (*@\HLJLcs{{\#}}@*) (*@\HLJLcs{this}@*) (*@\HLJLcs{means}@*) (*@\HLJLcs{contour}@*) (*@\HLJLcs{integral}@*)
\end{lstlisting}

\begin{lstlisting}
-1.2485382363935424 + 1.949326343919058im
\end{lstlisting}


\begin{lstlisting}
(*@\HLJLn{g}@*) (*@\HLJLoB{=}@*) (*@\HLJLn{im}@*)(*@\HLJLoB{*}@*)(*@\HLJLnf{Fun}@*)(*@\HLJLp{(}@*)(*@\HLJLn{t}@*)(*@\HLJLoB{->}@*) (*@\HLJLnf{f}@*)(*@\HLJLp{(}@*)(*@\HLJLnf{exp}@*)(*@\HLJLp{(}@*)(*@\HLJLn{im}@*)(*@\HLJLoB{*}@*)(*@\HLJLn{t}@*)(*@\HLJLp{))}@*)(*@\HLJLoB{*}@*)(*@\HLJLnf{exp}@*)(*@\HLJLp{(}@*)(*@\HLJLn{im}@*)(*@\HLJLoB{*}@*)(*@\HLJLn{t}@*)(*@\HLJLp{),}@*) (*@\HLJLni{0}@*) (*@\HLJLoB{..}@*) (*@\HLJLn{\ensuremath{\pi}}@*)(*@\HLJLoB{/}@*)(*@\HLJLni{2}@*)(*@\HLJLp{)}@*)
(*@\HLJLnf{sum}@*)(*@\HLJLp{(}@*) (*@\HLJLn{g}@*) (*@\HLJLp{)}@*)  (*@\HLJLcs{{\#}}@*) (*@\HLJLcs{this}@*) (*@\HLJLcs{is}@*) (*@\HLJLcs{standard}@*) (*@\HLJLcs{integral}@*)
\end{lstlisting}

\begin{lstlisting}
-1.2485382363935429 + 1.9493263439190578im
\end{lstlisting}


An important property of a contour is its \emph{arclength}:

\textbf{Definition (Arclength)} The \emph{arclength} of $\gamma$ is defined as

\[
    {\cal L}(\gamma) := \int_a^b |\gamma'(t)| dt
\]
A very useful result is that we can use the maximum and arclength to bound integrals:

\textbf{Proposition (ML)} Let $f : \gamma \rightarrow {\mathbb C}$ and 

\[
        M = \sup_{z \in \gamma} |f(z)|
\]
Then

\[
\left|\int_\gamma f(z) dz \right| \leq M {\cal L}(\gamma)
\]
\subsection{Cauchy's theorem}
\textbf{Proposition} If $f(z)$ is holomorphic on $\gamma$, then $\int_\gamma f'(z) dz = f(\gamma(b)) - f(\gamma(a))$

\textbf{Theorem (Cauchy)} If $f$ is holomorphic inside and on a closed contour $\gamma$, then  $\oint_\gamma f(z) dz = 0$

\subsubsection{Demonstration}
The following show that the contour of integration does not affect the integral for holomorphic functions:


\begin{lstlisting}
(*@\HLJLn{f}@*) (*@\HLJLoB{=}@*) (*@\HLJLnf{Fun}@*)(*@\HLJLp{(}@*) (*@\HLJLn{z}@*) (*@\HLJLoB{->}@*) (*@\HLJLnf{exp}@*)(*@\HLJLp{(}@*)(*@\HLJLn{z}@*)(*@\HLJLp{),}@*) (*@\HLJLnf{Arc}@*)(*@\HLJLp{(}@*)(*@\HLJLnfB{0.}@*)(*@\HLJLp{,}@*)(*@\HLJLnfB{1.}@*)(*@\HLJLp{,(}@*)(*@\HLJLni{0}@*)(*@\HLJLp{,}@*)(*@\HLJLn{\ensuremath{\pi}}@*)(*@\HLJLoB{/}@*)(*@\HLJLni{2}@*)(*@\HLJLp{)))}@*) (*@\HLJLcs{{\#}integrate}@*) (*@\HLJLcs{over}@*) (*@\HLJLcs{an}@*) (*@\HLJLcs{arc}@*)
(*@\HLJLnf{sum}@*)(*@\HLJLp{(}@*)(*@\HLJLn{f}@*)(*@\HLJLp{)}@*)  (*@\HLJLp{,}@*) (*@\HLJLnf{f}@*)(*@\HLJLp{(}@*)(*@\HLJLn{im}@*)(*@\HLJLp{)}@*)(*@\HLJLoB{-}@*)(*@\HLJLnf{f}@*)(*@\HLJLp{(}@*)(*@\HLJLni{1}@*)(*@\HLJLp{)}@*)
\end{lstlisting}

\begin{lstlisting}
(-2.1779795225909058 + 0.8414709848078968im, -2.177979522590906 + 0.8414709
848078968im)
\end{lstlisting}


\begin{lstlisting}
(*@\HLJLn{f}@*) (*@\HLJLoB{=}@*) (*@\HLJLnf{Fun}@*)(*@\HLJLp{(}@*) (*@\HLJLn{z}@*) (*@\HLJLoB{->}@*) (*@\HLJLnf{exp}@*)(*@\HLJLp{(}@*)(*@\HLJLn{z}@*)(*@\HLJLp{),}@*) (*@\HLJLnf{Segment}@*)(*@\HLJLp{(}@*)(*@\HLJLni{1}@*)(*@\HLJLp{,}@*)(*@\HLJLn{im}@*)(*@\HLJLp{))}@*) (*@\HLJLcs{{\#}}@*) (*@\HLJLcs{integrate}@*) (*@\HLJLcs{over}@*) (*@\HLJLcs{a}@*) (*@\HLJLcs{line}@*) (*@\HLJLcs{segment}@*)
(*@\HLJLnf{sum}@*)(*@\HLJLp{(}@*)(*@\HLJLn{f}@*)(*@\HLJLp{)}@*)  (*@\HLJLp{,}@*) (*@\HLJLnf{f}@*)(*@\HLJLp{(}@*)(*@\HLJLn{im}@*)(*@\HLJLp{)}@*)(*@\HLJLoB{-}@*)(*@\HLJLnf{f}@*)(*@\HLJLp{(}@*)(*@\HLJLni{1}@*)(*@\HLJLp{)}@*)
\end{lstlisting}

\begin{lstlisting}
(-2.1779795225909053 + 0.8414709848078966im, -2.177979522590907 + 0.8414709
848078968im)
\end{lstlisting}


The following demonstrates Cauchy's theorem, integration over a closed contour equals zero:


\begin{lstlisting}
(*@\HLJLn{f}@*) (*@\HLJLoB{=}@*) (*@\HLJLnf{Fun}@*)(*@\HLJLp{(}@*) (*@\HLJLn{z}@*) (*@\HLJLoB{->}@*) (*@\HLJLnf{exp}@*)(*@\HLJLp{(}@*)(*@\HLJLn{z}@*)(*@\HLJLp{),}@*) (*@\HLJLnf{Circle}@*)(*@\HLJLp{())}@*)  (*@\HLJLcs{{\#}}@*)  (*@\HLJLcs{Holomorphic!}@*)
(*@\HLJLnf{sum}@*)(*@\HLJLp{(}@*)(*@\HLJLn{f}@*)(*@\HLJLp{)}@*)
\end{lstlisting}

\begin{lstlisting}
2.9644937254112756e-17 - 1.4872544363724962e-16im
\end{lstlisting}


If \texttt{f} is not holomorphic, it doesn't apply:


\begin{lstlisting}
(*@\HLJLn{f}@*) (*@\HLJLoB{=}@*) (*@\HLJLnf{Fun}@*)(*@\HLJLp{(}@*) (*@\HLJLn{z}@*) (*@\HLJLoB{->}@*) (*@\HLJLnf{exp}@*)(*@\HLJLp{(}@*)(*@\HLJLn{z}@*)(*@\HLJLp{)}@*)(*@\HLJLoB{/}@*)(*@\HLJLn{z}@*)(*@\HLJLp{,}@*) (*@\HLJLnf{Circle}@*)(*@\HLJLp{())}@*)  (*@\HLJLcs{{\#}}@*) (*@\HLJLcs{Not}@*) (*@\HLJLcs{holomorphic}@*) (*@\HLJLcs{at}@*) (*@\HLJLcs{zero}@*)
(*@\HLJLnf{sum}@*)(*@\HLJLp{(}@*)(*@\HLJLn{f}@*)(*@\HLJLp{)}@*)
\end{lstlisting}

\begin{lstlisting}
-9.96316757509274e-16 + 6.283185307179588im
\end{lstlisting}


\begin{lstlisting}
(*@\HLJLn{f}@*) (*@\HLJLoB{=}@*) (*@\HLJLnf{Fun}@*)(*@\HLJLp{(}@*) (*@\HLJLn{z}@*) (*@\HLJLoB{->}@*) (*@\HLJLnf{real}@*)(*@\HLJLp{(}@*)(*@\HLJLnf{exp}@*)(*@\HLJLp{(}@*)(*@\HLJLn{z}@*)(*@\HLJLp{)),}@*) (*@\HLJLnf{Circle}@*)(*@\HLJLp{())}@*)  (*@\HLJLcs{{\#}}@*) (*@\HLJLcs{Not}@*) (*@\HLJLcs{holomorphic}@*) (*@\HLJLcs{anywhere!}@*)
(*@\HLJLnf{sum}@*)(*@\HLJLp{(}@*)(*@\HLJLn{f}@*)(*@\HLJLp{)}@*)
\end{lstlisting}

\begin{lstlisting}
-3.3420237696193494e-16 + 3.141592653589793im
\end{lstlisting}



\end{document}
